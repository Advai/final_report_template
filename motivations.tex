\section{Motivation}

Desirable qualities such as replication and consistency across large cloud-scale systems require the use of distributed consensus at some level. 
However, consensus is a very expensive procedure in terms of communication and round trip complexity. 
This results in most of the clusters running consensus algorithms in practice to be quite small, with larger clusters built on top of them that typically benefit from consistent reads that such consensus algorithms offer. 
This provides a degree of crash-fault tolerance, while minimizing the cost to latency and availability for users. 
For hyper-scalar cloud service providers, availability or client facing availability is the most commonly used metric for evaluating system effectiveness.

Raft was originally proposed in 2014 as an easier to understand and implement consensus algorithm when compared to Paxos.
 Despite those claims,  there are many similarities between the two protocols.
 Specifically, research has shown that there exists a formal mapping between the two under moderate assumptions using refinement mapping [8].
 The result of such work has given the ability to port specific optimizations from Paxos to Raft without sacrificing consensus requirements.
 This is crucial because there are many variations of Paxos, as it has been an active area of research for over twenty years now.
 In more recent years, new byzantine fault tolerant algorithms such as Tendermint, Casper FFG [7], and HotStuff [6] have utilized an immutable chain-based log structure which is a paradigm shift from the usual mutable log structures that Raft, Paxos, and PBFT use.
 Some less formal, raft-specific immutable log variants are also being explored in recent times [4].
  As a result, we believe that there can be much research needed in Raft optimizations(?)
